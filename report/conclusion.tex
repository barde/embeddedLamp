The project was done by four members which took responsibility concerning parts which were defined in the first mock-up design.
Everything was adapted to the Panda Board, which is a development platform for the OMAP4XXX chipset which is based on 
ARM.

The hardware was completely build by us. The user space programs for testing the bar showed the WLAN link quality as reported
by the wireless driver and another program showed the CPU load. Both scripts were done as bash-scripts which itself created 
heavy load.

An important fact are benchmarks which were done after finishing the project. We added another possibility of sending the data 
to the LED-bar. Firstly, a SPI interface done with a kernel protocol driver was done with configurable bus and data transfer speed.
Secondly, another form of sending data was done by using a micro controller which prepared and refreshed the LEDs.
Between this two there was no considerable load difference, even when increasing the bus speed of SPI to 10 MHz.

As an improvement the user space programs would need to be developed in the C-language or even better, in assembly code.
