The main idea behind this project was to use the \textit{Pandaboard} - an embedded development board - to output information on an external device, in this case a LED bar with 40 LED's. The goal was to create the drivers and to be able to use it universally. We wanted to create the possibility to write different applications for this system, without changing anything on the driver level of the \textit{Pandaboard}. 
Examples for so called \textit{User-space} application are measuring the quality of a connected WLAN, measuring the signal strength of a WLAN and also to output the \textit{Load} - CPU usage - of the \textit{Pandaboard}.

\section{Pandaboard}
The \textit{Pandaboard} is a low-power, low-cost single-board computer development platform based on the Texas Instruments OMAP44xx system-on-a-chip family.\footnote{More about the \textit{Pandaboard} can be found on \url{http://pandaboard.org/}  and  \url{http://en.wikipedia.org/wiki/PandaBoard}.} The board displayed in \ref{fig:pandaBoard_Alone} shows our setup with an installed SD card, on which the operating system has been placed. We have used \textit{Debian}\footnote{\textit{Debain} is a wide-spreaded operationg system with many derivates. It is commonly used as desktop but also server operating system. More information about \textit{Debian} can be found on the official website \url{http://debain.org}.} as our operating system. There are also a lot more of operating systems pre-configured for the use on a \textit{Pandaboard}.

\begin{figure}[H]
   \centering
   \includegraphics[width=0.8\textwidth]{img/Pandaboard_Alone.jpg}%
   \caption{Pandaboard with installed SD card}
   \label{fig:pandaBoard_Alone}%
\end{figure}

\newpage
\subsection{Features of the Pandaboard}
In the following table \ref{tab:pandaBoardTable} we will show the most important hardware features of the \textit{Pandaboard}. Especially the extensions are very interesting for realizing our project as we have to decide, which of the pins and which protocol we can use.

\begin{table}[h]
\centering
\begin{tabular}{| l |  p{7cm} |}
\hline
\textbf{Feature} & \textbf{Specification} \\ \hline
CPU & OMAP4430 dual-core 1 GHz ARM Cortex-A9 MPCore CPU \\ \hline
Memory & 1 GB low power DDR2 RAM \\ \hline
GPU & POWERVR™ SGX540 graphics core supporting all major API's including OpenGL® ES v2.0, OpenGL ES v1.1, OpenVG v1.1 and EGL v1.3 \\ \hline
Audio & Low power audio, 3.5"  Audio in/out \\ \hline
Storage & Full size SD/MMC card cage \\ \hline
LAN & Onboard 10/100 Ethernet \\ \hline
Wireless connections & 802.11 b/g/n, Bluetooth® v2.1 + EDR \\ \hline
Expansiona & General purpose expansion header (I2C, SPI, GPMC, USB, MMC, DSS, ETM) \\ \hline
Price & ca. 180 Euro \\
\hline
\end{tabular}
\caption{Pandaboard feature list}
\label{tab:pandaBoardTable}
\end{table}

\begin{figure}[H]
   \centering
   \includegraphics[width=0.8\textwidth]{img/Pandaboard_Setup.png}%
   \caption{Setup of a \textit{Pandaboard},  source: \url{http://pandaboard.org/node/223/}}
   \label{fig:pandaBoard_Setup}%
\end{figure}

\newpage
\section{LED-Bar}
The LED-bar contains 40 LED's, which will be connected to the extension connectors and steered by the \textit{Pandaboard}. A more detailed description of the  circuitry will be presented in the next chapter \textit{Hardware Design}. 
\begin{figure}[H]
   \centering
   \includegraphics[width=0.4\textwidth, angle=90]{img/LED-Bar.png}%
   \caption{LED-Bar with 40 LED's and level-shifter.}
   \label{fig:ledBar}%
\end{figure}

\section{Goal to achieve}
As we already described in the introduction we would like to be able to run some application on the \textit{Pandaboard} and steer the LED-bar with it. In figure \ref{fig:completeProject} we have connected the board to a micro controller and the micro controller to the LED-bar. This case was for testing the functionality. The final result should be running without the micro controller inbetween the board and the LED-bar.
\begin{figure}[H]
   \centering
   \includegraphics[width=0.8\textwidth]{img/Panda_and_LED_Bar.jpg}%
   \caption{Powering the LED-bar with the Pandaboard and a micro controller}
   \label{fig:completeProject}%
\end{figure}